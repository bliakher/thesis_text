\chapter*{Závěr}
\addcontentsline{toc}{chapter}{Závěr}

V rámci bakalářské práce byly analyzovány existující aplikace pro prohlížení úředních desek. Také byly prozkoumány možnosti, které nabízí zveřejňování dat z úředních desek jako otevřených dat podle OFN pro úřední desky. Na základě analýzy byly sestaveny požadavky na aplikaci, která by pracovala s daty z úředních desek zveřejněných v NKOD jako otevřená data a umožňovala by jejich vizualizaci a validaci.

Podle požadavků byl proveden návrh a následně implementace aplikace. Aplikace umožňuje přehledně prohlížet a vyhledávat informace na úředních deskách. Z pohledu poskytovatele dat stačí data zveřejnit v NKOD a není potřeba žádná další činnost proto, aby byla data vizualizována aplikací. Aplikace také poskytovatelům nabízí validaci dat podle specifikace OFN, včetně vysvětlení významu dat a řešení nejčastějších problémů se zveřejněním dat.

Aplikace propojuje data z úředních desek s informacemi o jejich poskytovatelích, získanými z dalších registrů otevřených dat, což umožňuje vizualizovat úřední desky na mapě a zobrazit statistický přehled poskytovatelů dat.

Aplikace byla otestována v uživatelském testování, kde ji uživatelé ohodnotili jako snadno použitelnou. Na základě podnětů od uživatelů byly vylepšené některé prvky v uživatelském rozhraní.

Aplikaci je možné dále vylepšovat. Některé návrhy na rozšíření byly popsané ve zpětné vazbě uživatelů k testování, \autoref{sub:vysledky-testovani}. 

Dalším možným rozšířením je vyhledávání a filtrování informací napříč všemi úředními deskami, nebo nějakou vybranou skupinou desek. Aplikace zatím umožňuje pouze vyhledávání informací v rámci jedné desky. Toto rozšíření v současné době neumožňuje architektura aplikace, postavená na straně klienta (viz \autoref{sec:architektura}), což je dané požadavky na způsob nasazení aplikace (viz \autoref{sub:tech-poz}). Vyhledávání ve velkém množství informací na straně klienta by bylo velmi neefektivní.