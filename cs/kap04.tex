
\chapter{Implementace}\label{kap:implementace}

V této kapitole bude popsána implementace aplikace, která byla provedena na základě návrhu aplikace v kapitole \ref{kap:navrh}. Aplikace je nasazená na adrese \url{https://bliakher.github.io/uredni_desky/}. Zdrojový kód aplikace je zveřejněný na platformě GitHub \footnote{\url{https://github.com/}} na URL \url{https://github.com/bliakher/uredni_desky} a jako příloha \ref{sub:repo1} v podobě snaphotu repozitáře ve formátu ZIP.

Kapitola se bude věnovat hlavně použitým technologiím a dalším implementačním rozhodnutím. Popis projektu včetně souborové struktury a jednotlivých tříd a rozhraní je možné najít ve vývojářské dokumentaci (\autoref{sec:developer-docs}).

\section{Použité technologie}

Jak bylo popsáno v návrhu, aplikace je implementovaná jako \textit{single-page application} na straně klienta. Pro implementaci byl použit programovací jazyk TypeScript v kombinaci s frameworkem React.

\subsection{TypeScript}

TypeScript \cite{TS} je silně typovaný programovací jazyk, který je nadstavbou jazyka JavaScript. TypeScript oproti jazyku JavaScript přidává možnost definovat typy a rozhraní a anotovat proměnné, parametry a návratové hodnoty metod, což umožňuje typovou kontrolu za kompilace. TypeScript je možné transpilovat do jazyka JavaScript, čímž se zajistí podpora na různých prohlížečích.

TypeScript byl pro implementaci projektu vybrán hlavně kvůli možnosti typování, čímž se snažíme vyhnout typovým chybám, které vznikají v jazyce JavaScript. Definovaní rozhraní jednotlivých částí aplikace navíc umožňuje lepší dokumentaci toku dat v aplikaci.

Aplikace využívá TypeScript verze 4.5.5 \footnote{\url{https://www.typescriptlang.org/docs/handbook/release-notes/typescript-4-5.html}}, který se transpiluje do ECMAScript verze 5 \cite{ecma5}

\subsection{React}

React\footnote{\url{https://reactjs.org/}} je deklarativní framework pro vývoj uživatelských rozhraní. Na základě studie od JetBrains \cite{JBStudie} se jedná o nejoblíbenější javascriptový framework pro vývoj uživatelských rozhraní. 

React využívá virtuální DOM s memoizací. V paměti si vytváří cache datových struktur, které tvoří aktuálně vyrenderovaný DOM, při aktualizaci vypočítá rozdíl nového a starého stavu a aktualizuje pouze ty části DOM stromu, které jsou ovlivněné změnou stavu. Selektivní renderování umožňuje větší výkon aplikace. \cite{ReactReconciliation}

React vytváří uživatelské rozhraní pomocí komponent. Komponenty jsou nezávislé jednotky, které si udržují vnitřní stav.\cite{ReactComponents} Na základě stavu se komponenta vyrenderuje podle příslušné šablony. Pro psaní šablon React využívá syntaxi JavaScript XML neboli JSX \cite{ReactJSX}, která umožňuje vytvářet HTML prvky v jazyce JavaScript. Díky tomu, mohou být šablony umístěné přímo v kódu.

Aplikace používá React ve verzi 17.0.2 \cite{Reactv17}.

\subsection{React Bootstrap}

Při implementaci byla také použitá knihovna React Bootstrap\footnote{\url{https://react-bootstrap.github.io/}} ve verzi 2.2.2, která obsahuje připravené React komponenty, sloužící pro ostylování aplikace. Jedná se hlavně o prvky pro pozicování, umožňující jednoduše vytvářet rozhraní, která se umí přizpůsobit různým velikostem displeje, a také ostylované řídící prvky, jako jsou tlačítka a formuláře.

Tato knihovna také nabízí velkou kolekci volně použitelných ikon, které je možné importovat jako komponenty.\footnote{\url{https://react-icons.github.io/react-icons}}

\section{Vývojové prostředí}

Vývoj probíhal v editoru Visual Studio Code \footnote{\url{https://code.visualstudio.com/}} s použitím rozšíření pro zvýraznění syntaxe jazyka TypeScript.

Pro instalaci a spravování knihoven byl použit správce balíčků npm \footnote{\url{https://www.npmjs.com/}} ve verzi 8.3.2. Sloužil také pro sestavení a spouštění aplikace pomocí šablony prostředí Create React App.

\subsection{Create React App}\label{sub:create-react}

Create React App\footnote{\url{https://www.npmjs.com/package/create-react-app}} je prostředí pro vývoj a sestavování aplikací v Reactu. Umí vygenerovat šablonu projektu, která podporuje TypeScript. Během vývoje aplikace je možné používat vývojový server, kdy se aplikace automaticky kompiluje a sestavuje po každé změně v kódu a spouští se v běhovém prostředí Node.js \footnote{\url{https://nodejs.org/en/about/}}.

Prostředí využívá kompilátor Babel \footnote{\url{https://babeljs.io/}} pro transpilaci TypeScriptu a JSX do JavaScriptu a Webpack \footnote{\url{https://webpack.js.org/}} pro zabalení kódu do balíčků se statickým obsahem.

Pro použití šablony Create React App je potřeba npm verze 5.6 nebo vyšší a Node.js verze 14 nebo vyšší.

\section{Verzování}

Při vývoji aplikace bylo použito verzování na platformě GitHub \footnote{\url{https://github.com/}}. Repozitář se zdrojovým kódem je dostupný na URL \\ \url{https://github.com/bliakher/uredni_desky} a jako příloha \ref{sub:repo1}. Stav repozitáře na moment odevzdání práce je označený tagem \texttt{odevzdani}.

GitHub je platforma podporující vývoj softwaru za pomoci verzovacího nástroje Git\footnote{\url{https://git-scm.com/}}, což je otevřený verzovací systém.

\section{Nasazení aplikace}

Aplikace je hostovaná pomocí služby GitHub Pages \footnote{\url{https://docs.github.com/en/pages/getting-started-with-github-pages/about-github-pages}} na adrese \url{https://bliakher.github.io/uredni_desky/}. V repozitáři na GitHub je nastavená větev projektu \textit{gh-pages}, ze které služba GitHub Pages bere obsah pro webovou aplikaci. Při aktualizaci větve se automaticky spustí proces, který nasadí novou verzi aplikace.

Pro sestavení aplikace pro nasazení je použit npm balíček gh-pages \footnote{\url{https://www.npmjs.com/package/gh-pages}}. Tento balíček vyrobí produkční build aplikace, který odešle do větve \textit{gh-pages}, což spustí publikaci na GitHub Pages.


