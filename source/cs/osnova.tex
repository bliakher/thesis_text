
Obecná - úvodní část

- úvod

- otevřená data v ČR 
    - co jsou otevřená data
    - NKOD
        - zveřejňování dat
        - sbírá data od různých poskytovatelů
        - nově úřední desky
    - OFN obecně
    - OFN pro úřední desky
    - RPP
    
- analýza existujících řešení
    - eDesky
    - mapa samosprávy
    
Informatická část

- analýza
    - uživatelé:
        - 2 části aplikace - vizualizace a validace
        - vizualizace:
            - široká veřejnost - IT laik
            - vizuálně pěkné, přehlednost, jednoduchost,
            - požadavky:
                - prohlížení desek
                - fitrování, vyhledávání desek
                - pohlížení informací na deskách, přílohy
        
        - validace:
            - poskytovatelé dat
                - laik - starosta obce
                    - má zájem ověřit, že všechno funguje
                - IT odborník 
                    - někdo, kdo přímo zveřejňuje, nastavuje data
                    - umí porozumět technickým detailům
                    - př. CORS hlavička
                - novinář - validita dat
            - požadavky:
                - obecně ujištění, že jsou data zveřejněna správně
                - dostupnost distribuce
                    - špatný odkaz
                    - CORS hlavička
                - data odpovídají OFN - doporučené atributy
                - vysvětlení, k čemu slouží atributy
                    - př. url informace - odkaz na web úřadu - podrobnosti k informaci
                    
    - use cases - jak zapisovat?
    
    - design - wireframes ??
        - architektura
    
- získávání dat
    - NKOD - SPARQL endpoint - získání metadat všech desek publikovaných podle OFN
    - získání distribuce 
        - stažení desky ze serveru poskytovatele
        - url z metadat
    - další info - typ poskytovatele 
        - z RPP - SPARQL endpoint
            - podle ičo - podrobné info
                - název
                - právní forma osoby
                

- implementace
    - SPA - jednoduché nasazení, hostování na GitHub Pages
    - TypeScript + React - proč?